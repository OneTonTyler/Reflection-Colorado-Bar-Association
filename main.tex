\documentclass[12pt]{article}
\usepackage[utf8]{inputenc}

\usepackage[margin=1in]{geometry}
\usepackage{lipsum}

\usepackage[backend=biber,style=ieee]{biblatex}
\addbibresource{sources.bib}

\usepackage{titling}
\newcommand{\subtitle}[1]{%
	\posttitle{%
		\par\end{center}
	\begin{center}\large#1\end{center}
	\vskip0.1em}}%

\usepackage[shortlabels]{enumitem}

\title{Reflection\\
Colorado Bar Association ESG Presentation}
\subtitle{PEGN 430A}
\author{Tyler Singleton}
\date{29 March 2022}

\begin{document}
\maketitle

\newpage
\setlength{\parindent}{0pt}

% --- Questions Section --- %
\textbf{Questions} \\

% Question 1
\textbf{1. Based on the presentation in class today, (a) what is Environmental Social Governance?  (b) List the corporate perspectives of the three components of the definition. (c) Select one perspective from each component that resonates with you, and describe why it resonates with you.} 

\begin{enumerate}[(a)]
    \item What is Environmental Social Governance? \\
    From the presentation, Environmental Social Governance (ESG) is the combination of reviewing a corporation's perspective under Environmental, Social, and Governance. This review questions what values are and takes into account their actions to achieve these values.
    
    \item Corporate perspectives. \\
    There are three corporate components that the perspectives fall into. The following list is taken directly from the presentation PowerPoint \cite{ESG_Reflection}.
    
        \begin{enumerate}[(1)]
            \item Environmental \\
                Perspectives in this category focus on climate change, GHG emissions, resource requirements and consumption, waste streams, and land impact.
            
            \item Social \\
                The social perspectives are a focus on working conditions, health and safety, local community impact, conflict regions, and employee relations and diversity.
            
            \item Governance \\
                This component houses perspectives on executive pay, corruption and bribery, lobbying, board diversity, and tax strategy.
            
            
        \end{enumerate}
    
    \item Perspectives that resonate.
        
        \begin{enumerate}[(1)]
            \item Environmental - Climate Change \\
            The climate change perspective under our environmental component resonates the strongest with me. This is because our climate has suffered a significant impact from us. It is no surprise that we see stronger storms, unpredictable seasons, and sea ice melting at a significantly fast rate. So focusing on this perspective is a very important aspect for me. 
            
            \item Social - Employee Diversity \\
            Employee diversity under our social component is an interesting topic. The reason this perspective resonates so strongly with my is due to its controversy. In recent news Harvard, MIT, and other Ivy League universities have been found to be discriminating enrollment based on ethnicity. This can also be seen in the work place when corporations disclose their diversity report to show that are inclusive. But when does inclusiveness start to become discriminatory? The controversial dialog on this topic interests me, so this perspective is what resonates strongest with me. 
            
            \item Governance - Tax Strategy \\
            Finally, the perspective of lobbying under our governance component resonates with me. Lobbying is popular topic and is described as ``legal bribery'' by some. I agree with this paradigm and against it. Because of that, this perspective strongly resonates with me. 
                
        \end{enumerate}
    
\end{enumerate}

% Question 2
\textbf{2.Provide a brief description of at least two topics that Joe Lima shared today that help you better understand the purpose and enforcement of environmental regulations.} 

\begin{enumerate}
    \item Science based initiative \\
    This topic of environmental regulations based in science helps me to understand the purpose and enforcement of environmental regulations. Environmental reviews is a part of permitting a project meets local, state, and federal regulations. This promotes corporations to be environmental conscious and strive to evolve with changing environmental conditions.
    
    \item Public opinion \\
    The public opinion is a large part in shaping the purpose and enforcement of environmental regulations. As described in the presentation \cite{ESG_Reflection}, public opinion shapes political and regulatory responses through popular support. Combined with media outlets giving greater coverage.
\end{enumerate}

% Question 3
\textbf{3. Define ``Green Washing''.  Provide an example of Green Washing (other than the one described in the presentation).  Note:  You will find several examples during shopping.} \\

 In the presentation, ``Green Washing,'' is shown to be corporations marketing their brands and/or items as environmentally friendly, but fail to adequately be environmentally conscious -- their main goal is change their public image. An example of this is the myth surrounding ``Clean Coal.'' This use to be a popular topic to persuade public opinion via misinformation that burning coal can be environmentally friendly. \\
 
 \textbf{4. Use a Carbon Footprint Calculator to calculate your personal carbon footprint. (a)  Describe in which areas could you lessen your carbon footprint.  (b) Upload the result as a PDF to this assignment CANVAS space.} \\
 
 My total carbon footprint added to 58 $CO_2/year$. Some improvements I could make would be to change my vehicle. I currently drive a Tacoma with about 20-22 mpg as my daily driver. I could adjust this to a more fuel efficient and possibly electric vehicle since Car Fuel accounts for majority of my travel carbon footprint. Next would be my electricity usage. Switching to more efficient appliances would assist in reducing my carbon footprint. More specifically, changing from a single large oven to a duel oven as it is common for me to leave stews in the oven for 6-8 hours. So reducing the amount of space needed to heat would help. Finally, I eat a lot of meat. I should reduce my consumption. 
 
 \textbf{5. List and describe the biggest takeaway for you from Joe's presentation.  How will you incorporate it in your life going forward?}\\
 
 My biggest takeaway from Joe's presentation is the topic of ``Green Washing.'' Corporations will do anything to garner a positive public image through misinformation and paid studies. It is a constant battle to stay an informed consumer and not to take advertising at face value. 

\newpage
\printbibliography

\end{document}
